\documentclass[10pt, a4paper]{article}

% --- Pacotes de Formatação de Texto e Idioma ---
\usepackage[utf8]{inputenc}
\usepackage[T1]{fontenc}
\usepackage[portuguese]{babel}
\usepackage{helvet} % Fonte similar a Arial (Requisito: Arial)
\renewcommand{\familydefault}{\sfdefault} % Define sans-serif como padrão

% --- Configuração das Margens e Espaçamento ---
\usepackage{geometry}
\geometry{
    top=2.5cm, 
    bottom=2.5cm, 
    left=2.5cm, 
    right=2.5cm
}
\usepackage{setspace}
\singlespacing % Requisito: Espaçamento simples

% --- Pacotes Matemáticos e Gráficos ---
\usepackage{amsmath}
\usepackage{amssymb}
\usepackage{graphicx}
\usepackage{subcaption}
\usepackage{float}
\usepackage{booktabs} % Tabelas bonitas
\usepackage[table,xcdraw]{xcolor}

% --- Pacotes para Código Fonte (Python) ---
\usepackage{listings}
\definecolor{codegreen}{rgb}{0,0.6,0}
\definecolor{codegray}{rgb}{0.5,0.5,0.5}
\definecolor{codepurple}{rgb}{0.58,0,0.82}
\definecolor{backcolour}{rgb}{0.95,0.95,0.92}

\lstdefinestyle{mystyle}{
    backgroundcolor=\color{backcolour},   
    commentstyle=\color{codegreen},
    keywordstyle=\color{magenta},
    numberstyle=\tiny\color{codegray},
    stringstyle=\color{codepurple},
    basicstyle=\ttfamily\footnotesize, % Fonte monoespaçada pequena
    breakatwhitespace=false,         
    breaklines=true,                 
    captionpos=b,                    
    keepspaces=true,                 
    numbers=left,                    
    numbersep=5pt,                  
    showspaces=false,                
    showstringspaces=false,
    showtabs=false,                  
    tabsize=2,
    language=Python
}
\lstset{style=mystyle}

% --- Bibliografia ---
\usepackage[style=apa, backend=biber]{biblatex}
\addbibresource{Recursos/referencias.bib} % Certifica-te que tens este ficheiro

% --- Hiperligações ---
\usepackage[hidelinks]{hyperref}

% --- Definição da Capa (IPBeja / ESTIG) ---
\newcommand{\customtitlepage}{
    \begin{titlepage}
        \centering
        \vspace*{0.5cm}
        
        % Logos
        \begin{figure}[h!]
            \centering
            % Ajusta o nome da imagem conforme necessário
            \includegraphics[width=6cm]{Recursos/Logos/LOGO_IPB} 
            \vspace{0.5cm}
        \end{figure}

        % Instituição
        {\large \textbf{INSTITUTO POLITÉCNICO DE BEJA} \par}
        {\large Escola Superior de Tecnologia e Gestão \par}
        \vspace{0.5cm}
        {\large \textbf{Mestrado em Engenharia de Segurança Informática} \par}
        {\large Criptografia e Criptanálise Aplicadas \par}

        \vspace{3cm}

        % Título do Trabalho
        {\Huge \textbf{Desenvolvimento de Aplicação de Cifra Simétrica} \par}
        \vspace{0.5cm}
        {\Large \textbf{Relatório Técnico – Componente 1} \par}
        
        \vspace{3cm}

        % Autores (Grupo de 2)
        \begin{minipage}{0.45\textwidth}
            \begin{flushleft} \large
                \textbf{Realizado por:}\\
                Rafael Narciso (n.º 24473)\\
                Hugo Diogo (n.º 18803)
            \end{flushleft}
        \end{minipage}
        \begin{minipage}{0.45\textwidth}
            \begin{flushright} \large
                \textbf{Docente:}\\
                Prof. Rui Miguel Silva
            \end{flushright}
        \end{minipage}

        \vfill

        % Data
        {\large Beja, dezembro de 2025 \par}
    \end{titlepage}
}

% ----------------------------------------------------------------------
% INÍCIO DO DOCUMENTO
% ----------------------------------------------------------------------
\begin{document}

% Capa
\customtitlepage

% Índice
\newpage
\thispagestyle{empty}
\tableofcontents
\newpage

% Reiniciar contagem de páginas
\setcounter{page}{1}

% ----------------------------------------------------------------------
% 1. INTRODUÇÃO
% ----------------------------------------------------------------------
\section{Introdução}
Este relatório descreve o desenvolvimento de uma aplicação em modo gráfico para a cifragem e decifragem de ficheiros, realizada no âmbito da unidade curricular de Criptografia e Criptanálise Aplicadas. 

O objetivo principal deste trabalho é consolidar os conhecimentos sobre criptografia simétrica através da implementação "de raiz" de algoritmos clássicos (Vigenère e Playfair) e da integração de bibliotecas para algoritmos modernos (DES e AES).

% ----------------------------------------------------------------------
% 2. ENQUADRAMENTO TEÓRICO E TECNOLÓGICO
% ----------------------------------------------------------------------
\section{Enquadramento Teórico e Tecnológico}

\subsection{Cifras Clássicas}
As cifras clássicas implementadas operam ao nível do carater (byte visível).
\begin{itemize}
    \item \textbf{Vigenère:} Uma cifra polialfabética que utiliza uma chave e uma tabela (Tabula Recta) para substituir caracteres.
    \item \textbf{Playfair:} Uma cifra de digramas que utiliza uma matriz $5 \times 5$ gerada a partir de uma chave.
\end{itemize}

\subsection{Cifras Modernas}
\begin{itemize}
    \item \textbf{DES (Data Encryption Standard):} Algoritmo de bloco de 64 bits.
    \item \textbf{AES (Advanced Encryption Standard):} Sucessor do DES, operando com blocos de 128 bits e chaves de 128, 192 ou 256 bits.
\end{itemize}

\subsection{Tecnologias Utilizadas}
A aplicação foi desenvolvida na linguagem \textbf{Python} devido à sua versatilidade na manipulação de \textit{strings} e \textit{bytes}. Para a interface gráfica utilizou-se a biblioteca \textbf{CustomTkinter}, e para as cifras modernas a biblioteca \textbf{Cryptography/PyCryptodome}.

% ----------------------------------------------------------------------
% 3. ARQUITETURA E IMPLEMENTAÇÃO
% ----------------------------------------------------------------------
\section{Arquitetura e Implementação}

\subsection{Estrutura da Aplicação}
A aplicação foi desenvolvida seguindo uma arquitetura modular que separa a interface gráfica da lógica criptográfica. O ficheiro principal, \texttt{gui.py}, implementa a interface utilizando a biblioteca \textbf{Tkinter}, gerindo os eventos do utilizador e invocando as classes específicas para cada cifra (\texttt{aes\_cipher.py}, \texttt{des\_cipher.py}, \texttt{playfair\_cipher.py} e \texttt{vigenere\_cipher.py}).

Esta separação permite que os algoritmos sejam testados independentemente da interface, facilitando a manutenção e a correção de erros ("debugging").

\subsection{Implementação das Cifras Clássicas}
As cifras clássicas foram implementadas sem recurso a bibliotecas externas de criptografia, manipulando diretamente os caracteres ASCII e as suas representações numéricas.

\subsubsection{Cifra de Vigenère}
A implementação da cifra de Vigenère (classe \texttt{VigenereCipher}) baseia-se na aritmética modular. A função \texttt{\_extend\_key} garante que a chave tem o mesmo comprimento que o texto a cifrar. A cifragem é realizada através da soma dos índices dos caracteres na tabela alfabética, módulo 26.

\begin{lstlisting}[caption={Lógica central da cifra de Vigenère}, label={lst:vigenere}]
def encrypt(self, plaintext):
    for i, char in enumerate(plaintext):
        if char.isalpha():
            row = ord(key[i]) - ord('A')
            col = ord(char) - ord('A')
            
            encrypted_char = self.table[row][col]
            ciphertext += encrypted_char
    return ciphertext
\end{lstlisting}

\subsubsection{Cifra de Playfair}
A cifra de Playfair (classe \texttt{PlayfairCipher}) exigiu uma implementação mais complexa devido à necessidade de gerar uma matriz $5 \times 5$ e processar digramas (pares de letras).
\begin{itemize}
    \item \textbf{Matriz:} A função \texttt{\_create\_matrix} remove duplicados da chave e preenche a matriz com o restante alfabeto, fundindo as letras 'J' e 'I' numa única posição para caber na grelha de 25 caracteres.
    \item \textbf{Preparação do Texto:} A função \texttt{\_prepare\_text} insere um caráter de enchimento ('X') caso existam letras repetidas num digrama ou se o texto tiver um comprimento ímpar.
\end{itemize}

\begin{lstlisting}[caption={Tratamento de digramas na cifra Playfair}, label={lst:playfair}]
if text[i] == text[i + 1]:  # Letras iguais no par
    prepared += 'X'         # Insere 'X'
else:
    prepared += text[i + 1]
\end{lstlisting}

\subsection{Implementação das Cifras Modernas}
Para as cifras modernas, utilizou-se a biblioteca \textbf{PyCryptodome}, garantindo uma implementação robusta e segura dos algoritmos padrão.

\subsubsection{Modo de Operação e Padding}
Tanto para o AES como para o DES, optou-se pelo modo de operação \textbf{CBC (Cipher Block Chaining)}. Este modo requer um Vetor de Inicialização (IV) aleatório para garantir que o mesmo texto simples produza criptogramas diferentes.
O \textit{padding} (preenchimento) é realizado segundo a norma PKCS7, através da função \texttt{pad()} da biblioteca, garantindo que os dados têm um tamanho múltiplo do bloco (8 bytes para DES, 16 bytes para AES).

\subsubsection{Gestão do Vetor de Inicialização (IV)}
Uma decisão crítica de implementação foi a forma de armazenamento do IV. Como o IV é necessário para a decifragem mas não é secreto, a aplicação concatena o IV (em binário) diretamente no início do ficheiro cifrado.

\begin{lstlisting}[caption={Cifragem de ficheiros com AES em modo CBC}, label={lst:aes}]
def encrypt_file(self, data):
    cipher = AES.new(self.key, AES.MODE_CBC)
    ct_bytes = cipher.encrypt(pad(data, AES.block_size))
    # Retorna o IV concatenado com o texto cifrado
    return cipher.iv + ct_bytes
\end{lstlisting}

Na operação de decifragem, a aplicação lê os primeiros bytes do ficheiro (8 para DES, 16 para AES) para recuperar o IV e inicializar o algoritmo para a decifragem do restante conteúdo.
% ----------------------------------------------------------------------
% 4. ANÁLISE DE SEGURANÇAx\
% ----------------------------------------------------------------------
\section{Análise de Segurança e Decisões de Projeto}

A implementação realizada cumpre os requisitos funcionais propostos, no entanto, uma análise de segurança rigorosa revela vulnerabilidades inerentes à natureza académica do projeto, bem como decisões de design tomadas para mitigar riscos nos algoritmos modernos.

\begin{enumerate}
    \item \textbf{Armazenamento de Material Criptográfico:} A leitura de chaves a partir de ficheiros de texto simples (\textit{cleartext}) viola princípios fundamentais de gestão de segredos. Num ambiente de produção, esta abordagem expõe o sistema a fugas de informação triviais. A solução adequada passaria pela utilização de cofres digitais (\textit{Key Management Systems} - KMS) ou módulos de segurança de hardware (HSM), garantindo que as chaves nunca residem em disco de forma legível.
    
    \item \textbf{Segurança Semântica e Gestão do IV:} Para os algoritmos AES e DES, optou-se pelo modo de operação CBC (\textit{Cipher Block Chaining}) em detrimento do inseguro modo ECB. Para garantir a segurança semântica — assegurando que a cifragem do mesmo texto simples resulta em criptogramas distintos —, é gerado um Vetor de Inicialização (IV) pseudoaleatório e criptograficamente seguro para cada operação. O IV é concatenado ao início do ficheiro cifrado; esta prática é segura, pois o IV não é um segredo, necessitando apenas de ser único (nonce) e imprevisível.
    
    \item \textbf{Integridade e Autenticidade (Limitação):} É importante notar que a aplicação garante a confidencialidade, mas não a integridade dos dados. O modo CBC é maleável a ataques de inversão de bits. A ausência de um mecanismo de autenticação de mensagem (como HMAC ou o uso de modos autenticados como GCM - \textit{Galois/Counter Mode}) significa que a aplicação não deteta se o ficheiro cifrado foi adulterado por terceiros antes da decifragem.
    
    \item \textbf{Gestão de Memória:} A utilização de Python, uma linguagem de gestão automática de memória, impede o controlo granular sobre a limpeza de variáveis sensíveis (chaves e texto simples) da RAM após o uso, o que poderia permitir ataques de \textit{memory dump} em máquinas comprometidas.
\end{enumerate}
% ----------------------------------------------------------------------
% 5. MANUAL DE UTILIZAÇÃO E TESTES
% ----------------------------------------------------------------------
\section{Manual de Utilização e Testes}
\subsection{Interface Gráfica}
A Figura \ref{fig:gui} demonstra a interface principal da aplicação.

\begin{figure}[h!]
    \centering
    % \includegraphics[width=0.8\textwidth]{caminho_para_print.png}
    \caption{Interface Gráfica da Aplicação}
    \label{fig:gui}
\end{figure}

\subsection{Casos de Teste}
Foram realizados testes com ficheiros de texto (ASCII) para as cifras clássicas e ficheiros binários (imagens PDF) para as cifras modernas.

% ----------------------------------------------------------------------
% 6. CONCLUSÃO
% ----------------------------------------------------------------------
\section{Conclusão}
O desenvolvimento deste projeto permitiu compreender as diferenças fundamentais entre a manipulação de texto em cifras clássicas e a manipulação de blocos de bits em cifras modernas. A implementação "de raiz" revelou a complexidade algoritmica por trás de métodos históricos.

% ----------------------------------------------------------------------
% REFERÊNCIAS
% ----------------------------------------------------------------------
\newpage
\printbibliography

\end{document}